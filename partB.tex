\begin{frame}
 \frametitle{}
 \tableofcontents
\end{frame}

\section{Classical diagram chases}


% What are diagram chases?
\begin{frame}[fragile]
 \frametitle{What are diagram chases?}
 \pause
 \begin{block}{}
  Diagram chases are a tool in homological algebra
  used for proving 
  \pause
  \begin{enumerate}
   \item properties
   \pause
   \item {\only<7->{\color{red}}the existence}
   \pause
  \end{enumerate}
  of morphisms 
  \pause
  situated in (commutative) diagrams of prescribed shape.
 \end{block}
\end{frame}

% Connecting homomorphism in the snake lemma
\begin{frame}[fragile]
 \frametitle{Connecting homomorphism in the snake lemma}
 \begin{center}
 \begin{tikzpicture}[mystyle/.style={scale=.7}]
  \matrix[matrix of math nodes,column sep={70pt,between origins},row sep={40pt,between origins}] (s)
  { & & & |[name=Kernel]| \text{\only<1>{$\kernel(\gamma)$}\only<2>{${\color{red}c} \in \kernel(\gamma)$}\only<3->{$c \in \kernel(\gamma)$ }} & \\
    &|[name=A]| A &|[name=B]| \text{\only<1-3>{$B$}\only<4>{${\color{red}b} \in B$}\only<5->{$b \in B$}} &|[name=C]| \text{\only<1-2>{$\coker\delta$}\only<3>{${\color{red}c} \in \coker\delta$}\only<4->{$c \in \coker\delta$}} &|[name=01]| 0 \phantom{C}  \\
    |[name=02]| \phantom{a'} 0 &|[name=A']| \text{ \only<1-5>{$\ker\lambda$}\only<6>{${\color{red}a'} \in \ker\lambda$}\only<7->{$a' \in \ker\lambda$} } &|[name=B']| \text{ \only<1-4>{$B'$}\only<5>{${\color{red}b'} \in B'$}\only<6->{$b' \in B'$} } &|[name=C']| C' \\
    & |[name=Coker]| \text{ \only<1-6>{$\coker(\alpha)$}\only<7>{${\color{red}a' + \im( \alpha )}\in \coker( \alpha )$}\only<8->{$a' + \im( \alpha )\in \coker( \alpha )$} } & & & \\
  };
              \path[->,thick] (Kernel) edge (C);
              \path[->,thick] (C) edge (01);
              \path[->,thick] (A) edge node[mystyle,anchor=south]{$\delta$} (B);
              \path[->,thick] (B) edge node[mystyle,anchor=south] {$\epsilon$} (C);
              \path[->,thick] (A) edge node[mystyle,anchor=east] {$\alpha$} (A');
              \path[->,thick] (B) edge node[mystyle, anchor=east] {$\beta$} (B');
              \path[->,thick] (C) edge node[mystyle,anchor=east] {$\gamma$} (C');
              \path[->,thick] (02) edge (A');
              \path[->,thick] (A') edge node[mystyle,anchor=north] {$\mu$} (B');
              \path[->,thick] (B') edge node[mystyle,anchor=south]{$\lambda$} (C');
              \path[->,thick] (A') edge (Coker);
              \visible<8->{
              \draw[->,rounded corners,thick,blue] (Kernel) -| ($(01.east)+(.5,0)$) |- ($(B)!.35!(B')$) -|
    ($(02.west)+(-.5,0)$) |- (Coker);}
  \end{tikzpicture}
 \end{center}
 
 \begin{block}{}
 \begin{center}
  \only<1>{Wanted: $\kernel(\gamma) \stackrel{\partial}{\longrightarrow} \coker( \alpha )$. \phantom{$\stackrel{:}{T_g}$} }
  \only<2>{Start: $c \in \kernel( \gamma )$. \phantom{$\stackrel{:}{T_g}$}}
  \only<3>{This lies in $\coker\delta$. \phantom{$\stackrel{:}{T_g}$}}
  \only<4>{Choose: $b \in \epsilon^{-1}(\{c\})$. \phantom{$\stackrel{:}{T_g}$}}
  \only<5>{Map: $b \stackrel{\beta}{\mapsto} b'$. \phantom{$\stackrel{:}{T_g}$} }
  \only<6>{Compute: $a' \in \mu^{-1}(b')$. \phantom{$\stackrel{:}{T_g}$} }
  \only<7>{Map: $a' \mapsto a' + \im( \alpha )$.\phantom{$\stackrel{:}{T_g}$}}
  \only<8-9>{ Result: $c \stackrel{{\color{blue}\partial}}{\mapsto} a' + \im( \alpha )$.
  \visible<9>{\textbf{Context}: modules} \phantom{$\stackrel{:}{T_g}$}}
  \end{center}
 \end{block}
\end{frame}


% Classical solutions: embedding theorems
\begin{frame}[fragile]
\frametitle{Classical solutions: embedding theorems}
 \pause
 \begin{block}{Freyd-Mitchell embedding theorem}
  \pause
  Any small abelian category $\mathbf{A}$ admits an exact fully faithful covariant embedding 
  \[
   F: \mathbf{A} \hookrightarrow R-\mathbf{mod}
  \]
  into the category of $R$-modules for some ring $R$.
 \end{block}
 \pause
 \begin{block}{Application: existence of morphisms}
  \begin{center}
   \begin{tikzpicture}
   \coordinate (r) at (4,0);
   \coordinate (d) at (0,-1);
   \node (A) {$\Hom_{\mathbf{A}}(A,B)$};
   \node (B) at ($(A) + (r)$) {$\Hom_{R-\mathbf{mod}}(FA,FB)$};
   \visible<6->{
   \node (a) at ($(A) + (d)$) {$F^{-1}\phi$};
   }
   \visible<5->{
   \node (b) at ($(B) + (d)$) {$\phi$};
   }
   
   \draw[draw=none] (A) -- node[]{$\cong$} (B);
   \visible<6->{
   \draw[draw=none] (a) -- node[rotate=90]{$\in$} (A);
   \draw[draw=none] (a) -- node[]{$\leftrightarrow$} (b);
   }
   \visible<5->{
   \draw[draw=none] (b) -- node[rotate=90]{$\in$} (B);
   }
  \end{tikzpicture}
  \end{center}
 \end{block}
 \visible<7->{
  Problem: this isomorphism between $\Hom$-sets is \textbf{not constructive}.
  }
\end{frame}


\section{Additive relations}

% Back to the snake lemma
\begin{frame}[fragile]
 \frametitle{Back to the snake lemma}
 \begin{center}
 \begin{tikzpicture}[mystyle/.style={scale=.7}]
  \matrix[matrix of math nodes,column sep={70pt,between origins},row sep={40pt,between origins}] (s)
  { & & & |[name=Kernel]| \text{{$c \in \kernel(\gamma)$ }} & \\
    &|[name=A]| A &|[name=B]| \text{{\only<2->{\color{red}}$b \in B$}} &|[name=C]| \text{{$c \in \coker\delta$}} &|[name=01]| 0 \phantom{C}  \\
    |[name=02]| \phantom{a'} 0 &|[name=A']| \text{ {$a' \in \ker\lambda$} } &|[name=B']| \text{ {$b' \in B'$} } &|[name=C']| C' \\
    & |[name=Coker]| \text{ {$a' + \im( \alpha )\in \coker( \alpha )$} } & & & \\
  };
              \path[->,thick] (Kernel) edge (C);
              \path[->,thick] (C) edge (01);
              \path[->,thick] (A) edge node[mystyle,anchor=south]{$\delta$} (B);
              \path[->,thick] (B) edge node[mystyle,anchor=south] {$\epsilon$} (C);
              \path[->,thick] (A) edge node[mystyle,anchor=east] {$\alpha$} (A');
              \path[->,thick] (B) edge node[mystyle, anchor=east] {$\beta$} (B');
              \path[->,thick] (C) edge node[mystyle,anchor=east] {$\gamma$} (C');
              \path[->,thick] (02) edge (A');
              \path[->,thick] (A') edge node[mystyle,anchor=north] {$\mu$} (B');
              \path[->,thick] (B') edge node[mystyle,anchor=south]{$\lambda$} (C');
              \path[->,thick] (A') edge (Coker);
              \draw[->,rounded corners,thick,blue] (Kernel) -| ($(01.east)+(.5,0)$) |- ($(B)!.35!(B')$) -|
    ($(02.west)+(-.5,0)$) |- (Coker);
  \end{tikzpicture}
 \end{center}
 
 \begin{block}{}
 \begin{center}
  \only<1>{ Result: $c \stackrel{{\color{blue}\partial}}{\mapsto} a' + \im( \alpha )$. \phantom{$\stackrel{:}{T_g}$} }
  \only<2>{ Crucial step: the \textbf{uncanonical} choice $b \in \epsilon^{-1}(\{c\})$. \phantom{$\stackrel{:}{T_g}$} }
  \only<3>{ Make this step canonical: \textbf{relations} instead of maps: $c \mapsto \epsilon^{-1}(\{c\})$ \phantom{$\stackrel{:}{T_g}$}}
  \end{center}
 \end{block}
\end{frame}

% Relations
\begin{frame}[fragile]
 \frametitle{Relations}
 Let $A,B$ be abelian groups.
 \pause
 \begin{block}{Definition}
  A subgroup $f \subseteq A \oplus B$ is called a \textbf{relation from $A$ to $B$}.
 \end{block}
 \pause
 \begin{block}{Example}
  Let $\epsilon: A \rightarrow B$ be a homomorphism of abelian groups. 
  \pause
  \[
   \{ (a,b) \in A \oplus B ~|~ \epsilon(a) = b \}
  \]
  is a relation from $A$ to $B$\pause, called \textbf{graph of $\epsilon$}\pause, and
  \[
   \epsilon^{-1} := \{ (b,a) \in B \oplus A ~|~ \epsilon(a) = b \}
  \]
  is a relation from $B$ to $A$\pause, called \textbf{pseudo-inverse of $\epsilon$}.
 \end{block}
\end{frame}

% Relations
\begin{frame}[fragile]
 \frametitle{Relations}
 \begin{block}{Composition of relations}
  \pause Given $f \subseteq A \oplus B$ and $g \subseteq B \oplus C$, define \pause
  \[
   g \circ f := \{ (a,c) \in A \oplus C ~|~ \exists b \in B: (a,b) \in f, (b,c) \in g \}
  \]
 \end{block}
 \pause
\begin{center}
 If $f$ and $g$ correspond to maps, this describes their usual composition.
\end{center}
\end{frame}

% Relations
\begin{frame}[fragile]
 \frametitle{Relations}
 \begin{itemize}
  \item[Q:] When does an additive relation $f \subseteq A \oplus B$ defines an honest map (a group homomorphism)?
 \end{itemize}
 \pause
 \begin{block}{Domain}
  $\domain(f) := \big\{ a \in A \mid \exists b \in B: (a,b) \in f \big\} \visible<5->{{\color{red}\ = A}}$
 \end{block}
 \pause
 \begin{block}{Defect}
  $\defect(f) := \big\{ b \in B \mid (0,b) \in f \big\} \visible<7->{{\color{red}\ = 0 }}$
 \end{block}
 \pause
 \begin{itemize}
  \item[A:] When it has a full domain \pause\pause and $0$ defect.
 \end{itemize}

\end{frame}

% The snake lemma with relations
\begin{frame}[fragile]
 \frametitle{The snake lemma with relations}
 
 \begin{center}
 \begin{tikzpicture}[mystyle/.style={scale=.7}]
  \matrix[matrix of math nodes,column sep={70pt,between origins},row sep={40pt,between origins}] (s)
  { & & & |[name=Kernel]| \kernel(\gamma) & \\
    &|[name=A]| A &|[name=B]| B &|[name=C]| \coker\delta &|[name=01]| 0 \phantom{C}  \\
    |[name=02]| \phantom{a'} 0 &|[name=A']| \ker\lambda &|[name=B']| B' &|[name=C']| C' \\
    & |[name=Coker]| \coker(\alpha) & & & \\
  };
              \only<1>{\path[->,thick] (Kernel) edge node[mystyle,anchor=west] {$\iota$} (C); }
              \only<2->{\path[->,blue,thick] (Kernel) edge node[mystyle,anchor=west] {$\iota$} (C); }
              
              \path[->,thick] (C) edge (01);
              \path[->,thick] (A) edge node[mystyle,anchor=south]{$\delta$} (B);
              
              \only<1->{\path[->,thick] (B) edge node[mystyle,anchor=south] {$\epsilon$} (C);}
              \only<3->{\path[->,red,thick, out=north west, in=north east] (C) edge node[mystyle,anchor=south] {$\epsilon^{-1}$} (B);}
              
              \path[->,thick] (A) edge node[mystyle,anchor=east] {$\alpha$} (A');
              
              \only<1-3>{\path[->,thick] (B) edge node[mystyle, anchor=east] {$\beta$} (B');}
              \only<4->{\path[->,blue,thick] (B) edge node[mystyle, anchor=east] {$\beta$} (B');}
              
              \path[->,thick] (C) edge node[mystyle,anchor=east] {$\gamma$} (C');
              \path[->,thick] (02) edge (A');
              
              \only<1->{\path[->,thick] (A') edge node[mystyle,anchor=north] {$\mu$} (B');}
              \only<5->{\path[->,red,thick, out=south west, in=south east] (B') edge node[mystyle,anchor=north] {$\mu^{-1}$} (A');}
              
              \path[->,thick] (B') edge node[mystyle,anchor=south]{$\lambda$} (C');
              
              \only<1-5>{\path[->,thick] (A') edge node[mystyle,anchor=east] {$\pi$}(Coker);}
              \only<6->{\path[->,blue,thick] (A') edge node[mystyle,anchor=east] {$\pi$}(Coker);}
              \visible<7->{
              \draw[->,blue,rounded corners,thick] (Kernel) -| ($(01.east)+(.5,0)$) |- ($(B)!.35!(B')$) -|
    ($(02.west)+(-.5,0)$) |- (Coker);}
  \end{tikzpicture}
 \end{center}

 \begin{block}{}
 \begin{center}
  \only<1>{Wanted: $\kernel(\gamma) \stackrel{\partial}{\longrightarrow} \coker( \alpha )$. \phantom{$\stackrel{:}{T_g}$} }
  \only<2-6>{
  \visible<6->{${\color{blue}\pi} \circ$ }
  \visible<5->{${\color{red}\mu^{-1}} \circ$ }
  \visible<4->{${\color{blue}\beta} \circ$ }
  \visible<3->{${\color{red}\epsilon^{-1}} \circ$} 
  \visible<2->{${\color{blue}\iota}$ \phantom{$\stackrel{:}{T_g}$}}
  }
  \only<7->{\textbf{$\partial$ is an honest map given by a composition of relations!} \phantom{$\stackrel{:}{T_g}$}}
  \end{center}
 \end{block}
 
\end{frame}

\section{Generalized morphisms}

% From relations to generalized morphisms
\begin{frame}[fragile]
 \frametitle{From relations to generalized morphisms}
 \begin{block}{}
 \begin{center}
  \begin{itemize}
   \item \textbf{\color{blue}Wanted}: a categorical framework for relations.
   \pause
   \item \textbf{\color{blue}Solution}: generalized morphisms.
  \end{itemize}
 \end{center}
 \end{block} 
\end{frame}

% From relations to generalized morphisms
\begin{frame}[fragile]
 \frametitle{From relations to generalized morphisms}
 Let $A,B$ be objects in an abelian category $\mathbf{A}$.
 \pause
 \begin{block}{\visible<2->{Relation} \visible<5->{$\rightsquigarrow$ generalized morphism} \visible<6->{(data structure: span)}}
  \begin{center}
   \begin{tikzpicture}[mystyle/.style={scale=.7}]
  \matrix[matrix of math nodes,column sep={40pt,between origins},row sep={40pt,between origins}] (s)
  {
    |[name=A]| \text{ \visible<4->{$A$} } & |[name=AplusB]| \text{ \visible<2-3>{$A \oplus B$} } & |[name=B]| \text{ \visible<4->{$B$} } \\
    & |[name=D]| D & \\
  }; 
  \visible<2>{
  \path[right hook->, thick] (D) edge (AplusB);
  }
  \visible<3>{
  \path[right hook->, thick] (D) edge node[mystyle,anchor=east] {$\left( \begin{array}{cc} {\color{red}\alpha} & {\color{blue}\beta} \end{array} \right)$} (AplusB);
  }
  \visible<4->{
  \path[->, thick, red] (D) edge node[mystyle, anchor= east]{ ${\color{red}\alpha}$ }(A);
  \path[->, thick, blue] (D) edge node[mystyle, anchor= west]{ ${\color{blue}\beta}$ }(B);
  }
  \visible<5->{
  \path[->, dashed, thick] (A) edge (B);
  }
 \end{tikzpicture}
  \end{center}
 \end{block}
 \pause
 \pause
 \pause
 \pause
 \pause
 \begin{block}{Equality}
  \pause
  Two spans $( \alpha, \beta )$ and $( \alpha', \beta' )$ are \textbf{equal as generalized morphisms}
  if \pause
  \[
   \im\left( ( \alpha, \beta ): D \rightarrow A \oplus B \right) = \im\left( ( \alpha', \beta' ): D' \rightarrow A \oplus B \right).
  \]
 \end{block}
\end{frame}

% Composition of generalized morphisms
\begin{frame}[fragile]
 \frametitle{Composition of generalized morphisms}
 
 \begin{block}{Composition} \pause
 \begin{overlayarea}{\linewidth}{7.3\baselineskip}
 \begin{center}
   \begin{tikzpicture}
      \matrix (s) [matrix of math nodes,column sep={40pt,between origins},row sep={40pt,between origins}]
      {  A &  & \phantom{B}  &  & C \\
         & \phantom{X}  &  & \phantom{Y}  &  & \\ 
         & & \phantom{X \times_B Y} & & \\ };
      
      \visible<2-6>{
	\path[->,dashed,thick] (s-1-1) edge (s-1-3);
      }
      \visible<7->{
	\path[->,color=gray,dashed,thick] (s-1-1) edge (s-1-3);
	\path[-,dashed,thick] (s-1-1) edge (s-1-3);
      }
      \path[->,dashed,thick] (s-1-3) edge (s-1-5);
	
      \visible<2-6>{
	\path[->,thick] (s-2-4) edge (s-1-5);  
	\path[->,thick] (s-2-2) edge (s-1-1);
      }
      
      \visible<2-3>{
	\path[->,thick] (s-2-2) edge (s-1-3);
	\path[->,thick] (s-2-4) edge (s-1-3);
	\node at (s-2-2) {$X$};
	\node at (s-2-4) {$Y$};
	\node at (s-1-3) {$B$};
      }
	
      
      \visible<4-6>{
	  \path[->,color=red,thick] (s-2-2) edge (s-1-3);
	  \path[->,color=red,thick] (s-2-4) edge (s-1-3);
	  \node[color=red] at (s-1-3) {$B$};
	  \node[color=red] at (s-2-2) {$X$};
	  \node[color=red] at (s-2-4) {$Y$};
      }
      
      
      \visible<5-6>{
	\path[->,color=red,thick] (s-3-3) edge (s-2-2);
	\path[->,color=red,thick] (s-3-3) edge (s-2-4);
	\node[color=red] at (s-3-3) {$X \times_B Y$};
      }
      
      \visible<7->{
	\node at (s-2-2) {$X$};
	\node at (s-2-4) {$Y$};
	\node[color=gray] at (s-1-3) {$B$};
	\node at (s-3-3) {$X \times_B Y$};
	\path[->,color=gray] (s-2-2) edge (s-1-3);
	\path[->,color=gray] (s-2-4) edge (s-1-3);
	\path[->,thick,thick] (s-2-4) edge (s-1-5);  
	\path[->,thick,thick] (s-2-2) edge (s-1-1);
	\path[->,thick,thick] (s-3-3) edge (s-2-2);
	\path[->,thick,thick] (s-3-3) edge (s-2-4);
      }
   \end{tikzpicture}
   \end{center}
 \end{overlayarea}
 \end{block}
 \visible<8->{
 \begin{center}
  $\rightsquigarrow$ Category of generalized morphisms $\G( \mathbf{A} )$
 \end{center}}
 
\end{frame}

% Pseudo-inverses
\begin{frame}[fragile]
  \frametitle{Pseudo-inverses}
 \begin{block}{Pseudo-inverses}
  \begin{center}
  \begin{tabular}{ccc}
   \begin{tikzpicture}[baseline=(s-2-2)]
      \matrix[matrix of math nodes,column sep={20pt,between origins},row sep={25pt,between origins}] (s)
      {
      A & & &  & B\\
      & \phantom{D} & & \phantom{D} & \\
      & & D  &  & \\
      };
         \visible<1>{
         \path[->, thick] (s-3-3) edge (s-1-1);
         \path[->, thick] (s-3-3) edge (s-1-5);
         }
         \visible<2->{
         \path[->,red, thick] (s-3-3) edge (s-1-1);
         \path[->,blue, thick] (s-3-3) edge (s-1-5);
         }
         \visible<3->{\path[<->,bend left, thick] (s-2-2) edge (s-2-4);}
         
         \path[->,dashed, thick] (s-1-1) edge (s-1-5);
    \end{tikzpicture}
    
    &
    \pause \pause \pause  {$\rightsquigarrow$} \pause
    &
      \begin{tikzpicture}[baseline=(s-2-2)]
      \matrix[matrix of math nodes,column sep={20pt,between origins},row sep={25pt,between origins}] (s)
      {
      B & & &  & A\\
      & \phantom{D} & & \phantom{D} & \\
      & & D  &  & \\
      };
         \path[->,blue, thick] (s-3-3) edge (s-1-1);
         \path[->,red, thick] (s-3-3) edge (s-1-5);
         \path[->,dashed, thick] (s-1-1) edge (s-1-5);
    \end{tikzpicture}
   
   
  \end{tabular}
\end{center}
 \end{block}
\end{frame}

% Honest morphisms
\begin{frame}[fragile]
 \frametitle{Honest morphisms}
 \pause
 \begin{block}{Honest morphisms}
  $\mathbf{A}$ embeds into $\G( \mathbf{A} )$:
  \pause
  \begin{center}
   \begin{tabular}{ccc}
    \begin{tikzpicture}[baseline=(base)]
        \coordinate (r) at (2.5,0);
        \coordinate (d) at (0,-2);
        \node (A) {$A$};
        \node (B) at ($(A) + (r)$) {$B$};
        \node (base) at ($(A) + (0,-0.1)$) {};
        \draw[->,thick,blue] (A) -- (B);
    \end{tikzpicture}
    
    &
    \pause {$\mapsto$} \pause
    &
      \begin{tikzpicture}[baseline=(s-2-2)]
      \matrix[matrix of math nodes,column sep={20pt,between origins},row sep={25pt,between origins}] (s)
      {
      A & & &  & B\\
      & \phantom{A} & & \phantom{A} & \\
      & & A  &  & \\
      };
         \path[->, thick] (s-3-3) edge node[left]{$\id_A$} (s-1-1);
         \path[->,blue, thick] (s-3-3) edge (s-1-5);
         \path[->,dashed, thick] (s-1-1) edge (s-1-5);
    \end{tikzpicture}
  \end{tabular}
  \end{center}
  \pause
  Generalized morphisms with such a representation are called \textbf{honest}.
 \end{block}
\end{frame}

% 
\begin{frame}[fragile]
 \frametitle{Honest morphisms}
 \begin{itemize}
  \item[Q:] When does $A \stackrel{\alpha}{\longleftarrow} D \stackrel{\beta}{\longrightarrow} B$ define an honest morphism?
 \end{itemize}
 \pause
 \begin{block}{Domain}
  $\domain(A \stackrel{\alpha}{\longleftarrow} D \stackrel{\beta}{\longrightarrow} B) := \im( \alpha ) \visible<5->{{\color{red}\ = A }}$
 \end{block}
 \pause
 \begin{block}{Defect}
  $\defect(A \stackrel{\alpha}{\longleftarrow} D \stackrel{\beta}{\longrightarrow} B) := \beta( \kernel( \alpha ) ) \visible<7->{{\color{red}\ = 0 }}$
 \end{block}
 \pause
 \begin{itemize}
  \item[A:] When it has a full domain \pause\pause and $0$ defect.
 \end{itemize}

\end{frame}

% Computing representatives
\begin{frame}[fragile]
 \frametitle{Computing representatives}
 \pause
 Given an honest generalized morphism in $\G( \mathbf{A} )$, compute the corresponding morphism in $\mathbf{A}$.
 \pause
 \begin{block}{}
  \begin{center}
   \begin{tikzpicture}[baseline=(s-2-2)]
      \matrix[matrix of nodes,column sep={20pt,between origins},row sep={20pt,between origins}] (s)
      {
        & & |[name=P]|\text{\visible<4-5>{{\only<5>{\color{gray}}$A\sqcup_D B$}}} & &\\
        & &   & &\\
      |[name=A]|$A$ & & &  & |[name=B]|$B$\\
      &  & &  & \\
      & & |[name=D]| \text{\only<1-4>{{\only<4>{\color{gray}}$D$}} \only<5-7>{$A \times_{A\sqcup_D B} B$}}  &  & \phantom{$A \times_{A\sqcup_D B} B$} \\
      };
         \visible<1-3,5-7>{
         \path[->, thick] (D) edge (B);
         }
         \visible<1-3,5>{
         \path[->, thick] (D) edge (A);
         }
         \visible<4>{
         \path[->, thick,gray] (D) edge (B);
         }
         \visible<4>{
         \path[->, thick,gray] (D) edge (A);
         }
         
         \visible<6>{
         \path[->,thick] (D) edge node[above, inner sep = 0em,rotate=-45]{$\sim$} (A);
         }
         \visible<7>{
         \path[->,thick] (A) edge node[above, inner sep = 0em,rotate=-45]{$\sim$} (D);
         }
         \visible<8>{
         \path[->,thick] (A) edge (B);
         }
         
         \visible<4>{
         \path[->,thick] (A) edge (P);
         \path[->,thick] (B) edge (P);
         }
         \visible<5>{
         \path[->,thick,gray] (A) edge (P);
         \path[->,thick,gray] (B) edge (P);
         }
         
         
    \end{tikzpicture}
  \end{center}
 \end{block}
\end{frame}

% Computing representatives
\begin{frame}[fragile]
 \frametitle{Computing representatives}
 Given an honest generalized morphism in $\G( \mathbf{A} )$, compute the corresponding morphism in $\mathbf{A}$.
 \begin{block}{}
  \begin{center}
   \begin{tikzpicture}[baseline=(s-2-2)]
      \matrix[matrix of nodes,column sep={20pt,between origins},row sep={20pt,between origins}] (s)
      {
        & & |[name=P]|\text{\visible<2,3>{{\only<3>{\color{gray}}$\Q$}}} & & \phantom{$A\sqcup_D B$}\\
        & &   & &\\
      |[name=A]|$\Q$ & & &  & |[name=B]|$\Q$\\
      &  & &  & \\
      & & |[name=D]| \text{\only<1,2>{{\only<2>{\color{gray}}$\Q^2$}} \only<3-5>{$\Q$}}  &  & \phantom{$A \times_{A\sqcup_D B} B$} \\
      };
         \visible<1>{
         \path[->, thick] (D) edge (B);
         \path[->, thick] (D) edge (A);
         }
         \visible<2>{
         \path[->, thick,gray] (D) edge (B);
         \path[->, thick,gray] (D) edge (A);
         }
         \visible<3-5>{
         \path[->, thick] (D) edge node[right, inner sep = 0.5em] {$2$} (B);
         }
         \visible<3>{
         \path[->, thick] (D) edge node[left, inner sep = 0.5em]{$1$} (A);
         }
         \visible<4>{
         \path[->,thick] (D) edge node[left, inner sep = 0.5em]{$1$} node[above, inner sep = 0em,rotate=-45]{$\sim$} (A);
         }
         \visible<5>{
         \path[->,thick] (A) edge node[left, inner sep = 0.5em]{$1$} node[above, inner sep = 0em,rotate=-45]{$\sim$} (D);
         }
         \visible<6>{
         \path[->,thick] (A) edge node[above]{$2$} (B);
         }
         
         \visible<2>{
         \path[->,thick] (A) edge node[left, inner sep = 0.5em] {$2$} (P);
         \path[->,thick] (B) edge node[right, inner sep = 0.5em] {$1$}(P);
         }
         \visible<3>{
         \path[->,thick,gray] (A) edge node[left, inner sep = 0.5em,gray] {$2$} (P);
         \path[->,thick,gray] (B) edge node[right, inner sep = 0.5em,gray] {$1$}(P);
         }
         
         \visible<1>{
         \node[above,font=\tiny] at ($(A) + (0,-1.2)$) {$\left(\begin{array}{cc} 1 \\ -1 \end{array} \right)$};
         \node[above,font=\tiny] at ($(B) + (0,-1.2)$) {$\left(\begin{array}{cc} 2 \\ -2 \end{array} \right)$};
         }
         \visible<2>{
         \node[above,font=\tiny,gray] at ($(A) + (0,-1.2)$) {$\left(\begin{array}{cc} 1 \\ -1 \end{array} \right)$};
         \node[above,font=\tiny,gray] at ($(B) + (0,-1.2)$) {$\left(\begin{array}{cc} 2 \\ -2 \end{array} \right)$};
         }
         
         
    \end{tikzpicture}
  \end{center}
 \end{block}
\end{frame}

% Constructive diagram chases
\begin{frame}[fragile]
 \frametitle{Constructive diagram chases}
 \pause
 \begin{block}{Strategy for constructive diagram chases}
  \pause
  \begin{enumerate}
   \item Compute in $\G( \mathbf{A} )$ using pseudo-inverses and compositions.
   \pause
   \item Compute the honest representative of the resulting generalized morphism.
  \end{enumerate}
 \end{block}
\end{frame}

\begin{frame}[fragile]
 \frametitle{Example: functoriality of homology}
 Let $\left( P_\bullet, \partial \right)$ be a complex in an abelian category $\mathcal{A}$. \pause Then we can compute the generalized
 embedding of the $i$-th homology. \pause
 \begin{block}{}
 \begin{center}
   \begin{tikzpicture}
      \matrix (s) [matrix of math nodes,column sep=20pt,row sep=20pt,nodes in empty cells]
      {  P_{i+1} & & P_i & & P_{i-1} \\
         & \phantom{\Image \left( \partial_{i+1} \right)} & & \phantom{\Kernel \left( \partial_i \right)} & \phantom{\CH_i \left( P_\bullet \right)} \\  };
      \path[->,thick] (s-1-1) edge node[above]{$\partial_{i+1}$} (s-1-3);
      \path[->,thick] (s-1-3) edge node[above]{$\partial_{i}$} (s-1-5);
      
      \visible<4->{
        \node at (s-2-2) {$\Image \left( \partial_{i+1} \right)$};
        \path[right hook->,thick] (s-2-2) edge (s-1-3);
      }
      
      \visible<5->{
        \node at (s-2-4) {$\Kernel \left( \partial_i \right)$};
      }
      
      \visible<5-7>{
        \path[right hook->,thick] (s-2-4) edge (s-1-3);
      }
      
      \visible<6->{
        \path[right hook->,thick] (s-2-2) edge (s-2-4);
      }
      
      \visible<7->{
        \node at (s-2-5) {$\CH_i \left( P_\bullet \right)$};
      }
      
      \visible<7>{
        \path[->>,thick] (s-2-4) edge (s-2-5);
      }
      
      \visible<8->{
        \path[->>,color=red,thick] (s-2-4) edge (s-2-5);
        \path[right hook->,color=red,thick] (s-2-4) edge (s-1-3);
        \path[left hook->,dashed,thick] (s-2-5) edge (s-1-3);
      }
      
   \end{tikzpicture}
  \end{center}
 \end{block}
\end{frame}

\begin{frame}[fragile]
 \frametitle{Example: functoriality of homology}
 
 \begin{theorem}
  Let $\mathcal{A}$ be an abelian category and $\varepsilon: P_\bullet \rightarrow Q_\bullet$ a chain morphism. \pause
  Then the morphism $\CH_i \left( P_\bullet \right) \rightarrow \CH_i \left( Q_\bullet \right)$ can be computed using generalized morphisms: \pause
  \begin{center}
   \begin{tikzpicture}
      \matrix (s) [matrix of math nodes,column sep=20pt,row sep=20pt,nodes in empty cells]
      {  \phantom{\CH_i \left( P_\bullet \right)} & P_i & Q_i & \phantom{ \CH_i \left( Q_\bullet \right) } \\ };
      
      \path[->,thick] (s-1-2) edge node[above] {$\varepsilon_i$} (s-1-3);
      
      \visible<4->{
        \node at (s-1-1) {$\CH_i \left( P_\bullet \right)$};
        \path[->,dashed,thick] (s-1-1) edge (s-1-2);
      }
      
      \visible<5->{
        \node at (s-1-4) {$\CH_i \left( Q_\bullet \right)$};
      }
      \visible<5>{
        \path[->,dashed,thick] (s-1-4) edge (s-1-3);
      }
      
      \visible<6->{
        \path[->,dashed,color=red,thick] (s-1-3) edge (s-1-4);
      }
      \visible<7->{
        \path[->, bend right,thick] (s-1-1) edge (s-1-4);
      }
   \end{tikzpicture}
  \end{center}
 \end{theorem}
\end{frame}

\begin{frame}[fragile]
  \frametitle{Demo}
    Snake lemma with generalized morphisms for
    \begin{itemize}
      \item vector spaces,
      \item abelian groups.
    \end{itemize}
 \end{frame}